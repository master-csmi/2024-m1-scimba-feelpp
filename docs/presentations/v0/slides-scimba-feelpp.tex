

\documentclass[10pt]{beamer}
\usetheme{Warsaw}
\usepackage[T1]{fontenc}
\usepackage[utf8]{inputenc}
\usepackage{chronosys}
\usepackage{hyperref}

\title{ScimBa Feel++}
\author{ Helya Amiri \\ Rayen Tlili }
\date{}

\begin{document}
\frame{\titlepage}
\begin{frame}
    \tableofcontents
\end{frame}

% Partie 1: Introduction


\section{Introduction to the project}
\begin{frame}
\frametitle{Introduction}
This project aims for the coupling of ScimBa and Feel++. This involves integrating the capabilities of both tools. 
\\This integration would allow users to leverage the advanced features of each of these libraries and use them synergistically to solve complex problems.



\end{frame}



% Partie 2: 
\section{Presentation of the libraries}
\begin{frame}
\setbeamertemplate{blocks}[rounded][shadow=false]

\setbeamertemplate{blocks}[rounded][shadow=false]
\frametitle{ScimBa}

ScimBa is a project that gathers tools for scientific machine learning, particularly focusing on tools for solving partial differential equations (PDEs) and other related tasks in scientific computing.

 \begin{itemize}
        \item ScimBa provides a platform for integrating machine learning techniques with traditional scientific computing methods. 
        
        %This integration enables researchers to develop sophisticated models and algorithms for solving complex scientific problems.

        \item One of the core features of ScimBa is its collection of solvers for PDEs. 
        
        %These solvers are essential for simulating and analyzing physical processes in fields such as fluid dynamics, heat transfer, and quantum mechanics.
    \end{itemize}
\end{frame}
\begin{frame}
\setbeamertemplate{blocks}[rounded][shadow=false]
\frametitle{Feel++}

Feel++ is an implementation in C++ that integrates Galerkin methods, encompassing both finite element methods and spectral element methods, to address partial differential equations 1D, 2D and 3D domains.

 \begin{itemize}
        \item Feel++ comes with a set of toolboxes for solving various physics-based problems, including fluid mechanics, solid mechanics, heat transfer ... 
        
        % These toolboxes provide pre-built applications and libraries for solving specific types of problems.



        \item Feel++ offers a Python interface (pyFeel++) that allows users to manipulate mathematical objects and solve PDEs using Python. 
        
        %This provides flexibility and interoperability with other Python-based libraries and tools. 
        
        
    \end{itemize}
\end{frame}


% Partie 3:

\section{Approach}

\begin{frame}
\setbeamertemplate{blocks}[rounded][shadow=false]
\frametitle{Coupling}
\begin{block}{Tight Coupling}
Modules are highly dependent on each other. Changes in one module often require corresponding changes in other modules.

%This type of coupling can make the codebase rigid and difficult to maintain, as modifications in one part of the code may have unintended consequences elsewhere..
\end{block}

\begin{block}{Loose Coupling}
Modules are relatively independent and have minimal dependencies on each other.

%This type of coupling promotes modularity and flexibility, as changes in one module are less likely to affect other modules.

\end{block}
\end{frame}



\begin{frame}
\setbeamertemplate{blocks}[rounded][shadow=false]
\frametitle{Starting Points}

\textbf{Objective:}

Implement loose coupling between ScimBa and Feel++ to leverage the strengths of both libraries effectively.



 \begin{itemize}
        \item Analyze the functionalities of both ScimBa and Feel++ to identify potential points of integration. 
        
        % Determine where the two libraries can interact with minimal dependencies, promoting loose coupling.



        \item Leverage Feel++'s capabilities for numerical simulations and scientific computing to generate diverse and realistic datasets. 
        
        % Implement methods to inject data generated by Feel++ into ScimBa's machine learning algorithms. 

        \item Define a clear interface and protocols to ensure communication between the two libraries. 
        
        %Verify that the loose coupling approach allows for seamless communication and interoperability between the two libraries.
        
        
    \end{itemize}   

\end{frame}



\section{References}
\begin{frame}
    \setbeamertemplate{blocks}[rounded][shadow=false]
    \frametitle{References}
    \begin{itemize}

        \item \href{https://docs.feelpp.org/user/latest/index.html}{Feel++ Documentation}
        
        \item \href{https://sciml.gitlabpages.inria.fr/scimba/}{ScimBa Documentation}

        \item \href{https://en.wikipedia.org/wiki/Coupling_(computer_programming)}{Coupling}
            
    \end{itemize}

\end{frame}

\end{document}
