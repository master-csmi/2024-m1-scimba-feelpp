\documentclass{article}
\usepackage{stmaryrd}
\usepackage{url}

\title{Project report: Coupling ScimBa and Feel++}

\author{Helya Amiri, Rayen Tlili}
\date{}

\begin{document}

\maketitle

\tableofcontents
\newpage


\section{Introduction}

This report presents the objectives, approach, and roadmap for the coupling of ScimBa and Feel++ libraries. ScimBa is a project aimed at integrating machine learning techniques with traditional scientific computing methods, while Feel++ is a C++ implementation of Galerkin methods for solving partial differential equations (PDEs). The coupling of these two libraries is expected to enhance their capabilities and enable researchers to solve complex scientific problems more effectively.



\subsection{Objectives}

This project seeks to facilitate the coupling of ScimBa and Feel++.
The primary objective is to establish loose coupling between ScimBa and Feel++, enabling efficient utilization of their respective strengths.

\subsection{Roadmap}

The roadmap for the project includes the following milestones:

\begin{enumerate}
    \item \textbf{Analysis and Planning:} Analyze the functionalities of ScimBa and Feel++ to identify integration points. Develop a plan for implementing loose coupling between the two libraries.
    \item \textbf{Implementation:} Implement loose coupling between ScimBa and Feel++ according to the planned approach. Develop methods for generating datasets using Feel++ and integrating them into ScimBa.
    \item \textbf{Testing and Validation:} Test the integrated system to ensure that communication between ScimBa and Feel++ is seamless. Validate the effectiveness of the coupling by solving complex scientific problems.
    \item \textbf{Documentation and Reporting:} Document the integration process, including interface definitions and communication protocols. Prepare a final report summarizing the project outcomes and lessons learned.
\end{enumerate}





\section{Bibliography}

\begin{itemize}
    \item Feel++ Documentation: \url{https://docs.feelpp.org/user/latest/index.html}
    \item ScimBa Documentation: \url{https://sciml.gitlabpages.inria.fr/scimba/}
    \item Coupling (Computer Programming): \url{https://en.wikipedia.org/wiki/Coupling_(computer_programming)}
    \item Using feel++:
    \url{https://www.cemosis.fr/events/course-solving-pdes-with-feel/}
\end{itemize}

\end{document}
